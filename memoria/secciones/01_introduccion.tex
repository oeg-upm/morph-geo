\chapter{Introducción}
%%---------------------------------------------------------
La introducción del TFG debe servir para que los profesores que evalúan el Trabajo puedan comprender el contexto en el que se realiza el mismo, y los objetivos que se plantean.

\section{Objetivos}
El objetivo principal del trabajo es introducir soporte al formato GeoPackage en herramientas de Linked Data
Geográfico desarrolladas por el Grupo de Ingeniería Ontológica.
En el OEG se ha venido tradicionalmente trabajando con el Instituto Geográfico Nacional para la exportación de algunos de sus datos geográficos a formato Linked Data. Un ejemplo se puede encontrar en https://datos.ign.es/

Recientemente, el Open Geospatial Consortium ha publicado el formato GeoPackage, que tiene el objetivo de convertirse en un estándar para la representación de datos geográficos. El objetivo de este trabajo es el de dar soporte GeoPackage para las herramientas normalmente utilizadas para este tipo de tareas.

\begin{itemize}
    \item Dar soporte GeoPackage a la herramienta Map4RDF.
    \item Dar soporte GeoPackage a la herramienta GeoKettle y su plugin para transformar a RDF.
    \item Realizar un procesado completo de todos los datos del IGN para generar este tipo de formato.
\end{itemize}

\section{Estado del Arte}

\subsection{Datos Geoespaciales}
Explicar lo que son en general, GIS, formatos...
Qué es el OGC y por qué importan sus estándares
\subsubsection{Shapefile}
El formato ESRI Shapefile (SHP) es un formato de archivo de datos espaciales desarrollado por la compañía ESRI.
A pesar de ser propietario, la especificación es abierta, y se considera un estandar de facto.
\subsubsection{GeoPackage}
Explicar las ventajas que tiene frente al shapefile para aclarar
por qué merece la pena añadir el soporte a las herramientas.

\subsection{Datos enlazados}
Explicar qué son los datos enlazados y en específico RDF

\subsection{Portales de Datos abiertos}
Poner ejemplo del IGN.

\subsection{Map4RDF}

\subsection{GeoKettle}
Explicar qué es GeoPackage y su plugin.

\section{Herramientas de desarrollo}
Docker

La imagen de docker de GeoKettle


%%---------------------------------------------------------
